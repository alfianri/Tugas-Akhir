\clearpage

\begin{minipage}{\textwidth}

\centering
	\singlespacing{
	\textbf{\textit{DRIVER DROWSINESS DETECTION USING
CONVOLUTIONAL NEURAL NETWORKS}}\\
 
 \vspace{1em}
	\mbox{Alfianri Manihuruk (120450088)}\\
	\textbf{Advisor I: Tirta Setiawan, S.Pd., M.Si}\\
	\textbf{Advisor II: Riksa Meidy Karim, S.Kom., M.Si., M.Sc}\\
  
}
\end{minipage}

\vspace{1em}

%\chapter*{ABSTRAK}
\normalsize \bfseries \centering \MakeUppercase{Abstract}
\phantomsection% 
\addcontentsline{toc}{chapter}{Abstract}
\\[2\baselineskip]

%taruh abstrak bahasa inggris di sini
\justifying \normalfont \normalsize

{

\textit{Traffic accidents caused by human negligence, such as drowsiness while driving, are a serious problem because these events involve other vehicles or other drivers on the road. Human negligence such as drowsiness can lead to accidents that have material and non-material costs. Many drivers ignore drowsiness, they still force themselves to drive when they should be resting. This research proposes a driver drowsiness detection system based on Convolutional Neural Network (CNN) to detect drowsiness. The parameters of Eye Aspect Ratio (EAR) and Mouth Aspect Ratio (MAR) are used to determine the class. There are three classes trained using CNN to identify driver patterns, such as 'sleepy and yawning', 'sleepy not yawning' and 'yawning not sleepy'. The three classes 
are a combination of EAR \& MAR parameters. The results show that this system is able to detect drowsiness with high accuracy. In this study, the best results were obtained by using the parameters \textit{learning rate} 0.0001, \textit{activation} ReLu and \textit{Optimizer} SGD. With system performance for accuracy, precision, recall, and F1-Score of respectively 92.97\%, 93.32\%, 92.72\%, and 92.98\%, respectively.
}

}

\textbf{Keyword}: \textit{Convolutional Neural Network, Drowsiness, Eye Aspect Ratio, Mouth
Aspect Ratio}

\clearpage